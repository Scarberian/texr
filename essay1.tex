
%----------------------------------------------------------------------------------------
%	PACKAGES AND OTHER DOCUMENT CONFIGURATIONS
%----------------------------------------------------------------------------------------

\documentclass[12pt]{article} % Default font size is 12pt, it can be changed here
\usepackage{times}
\usepackage{geometry} % Required to change the page size to A4
%\usepackage{datetime} %required to set arbitrary dates
%\newdate{date}{04}{17}{2013} %set arbitrary date M/D/Y
%\date{\displaydate{date}}
\geometry{a4paper} % Set the page size to be A4 as opposed to the default US Letter

%\title{\textsc{TITLE}}
%\author{\textsc{YOUR NAME}}
%\usepackage{fancyhdr}
%\setlength{\headheight}{15pt}

\rhead{\textsc{YOUR HEADER NAME}}
\lhead{\textsc{HEADER TITLE}


\linespread{1.6} % Line spacing
%\usepackage{scrpage2}
%\setlength\parindent{0pt} % Uncomment to remove all indentation from paragraphs
%\pagestyle{scrheadings}
%\ohead{Brian McElwain}
%\graphicspath{{./Pictures/}} % Specifies the directory where pictures are stored
\usepackage[longnamesfirst,nonamebreak]{natbib}
\renewcommand\refname{Bibliography}
\begin{document}
\pagenumbering{gobble}
\begin{titlepage}

%\newcommand{\HRule}{\rule{\linewidth}{0.5mm}} % Defines a new command for the horizontal lines, change thickness here

\center % Center everything on the page
 
%----------------------------------------------------------------------------------------
%	HEADING SECTIONS
%----------------------------------------------------------------------------------------




%\HRule \\[0.4cm]
\vspace*{5cm}
{ \huge \bfseries \textsc{YOUR ESSAY TITLE}}\\[0.4cm] % Title of your document
%\HRule \\[1.5cm]
 
%----------------------------------------------------------------------------------------
%	AUTHOR SECTION
%----------------------------------------------------------------------------------------

%\begin{minipage}{0.4\textwidth}
%\begin{flushleft} \large
%\emph{Author:}\\
%Brian \textsc{McElwain} % Your name
%\end{flushleft}
%\end{minipage}
~
%\begin{minipage}{0.4\textwidth}
%\begin{flushright} \large
%\emph{Supervisor:} \\
%Dr. James \textsc{Smith} % Supervisor's Name
%\end{flushright}
%\end{minipage}\\[4cm]

% If you don't want a supervisor, uncomment the two lines below and remove the section above



\vfill

\begin{flushright}
  \begin{tabular}{ l  c  r }
  \textsc{NAME HERE}\\ % Your name
\textsc{\#XXXXXXXXX}\\ % student number
\textsc{XXXXXXXXX}\\ % Course Code
\textsc{Dr.Foobar}\\ % Professor
DD/MM/YY
%{\today}\\ %dynamic dating
  \end{tabular}
\end{flushright}


%----------------------------------------------------------------------------------------
%	DATE SECTION
%----------------------------------------------------------------------------------------




 
%----------------------------------------------------------------------------------------

 % Fill the rest of the page with whitespace

\end{titlepage}
%----------------------------------------------------------------------------------------
%	TABLE OF CONTENTS
%----------------------------------------------------------------------------------------

%\tableofcontents % Include a table of contents

%\newpage % Begins the essay on a new page instead of on the same page as the table of contents 

%----------------------------------------------------------------------------------------
%	INTRODUCTION
%----------------------------------------------------------------------------------------
\pagestyle{fancy}
\pagenumbering{arabic}



%------------------------------------------------
% While pursuing a lofty goal there came a critical moment when there was light at the end of the tunnel. 
%But there it was entirely possible that it was not daylight but the light of an oncoming train.
After the confederation of Canada there were great challenges yet to be overcome and overcoming some of these challenges would be critical to Canada's survival.
In mountaineering they call it the crux, the hardest part of the climb. 
After confederation in 1867, Canada had just embarked on its ascent. 
Whether it would endure beyond a decade was not assured. 
The Red River uprising in the west challenged federal authority.
All the while, Americans were weighing the viability of northern annexation.
The resurgence of anti-confederate sentiments in Nova Scotia threatened national unity.
At the crux of these matters was Sir John A. Macdonald.
%------------------------------------------------

In Red River, a rebellious sentiment had taken root in the minds of the M\'{e}tis. 
Led by Louis Riel, they seized control of the settlement. 
He imposed himself as the head of a provisional government and compiled a list of demands and greivances in need of redress. 
Such acts put the territory in jeopardy of annexation by America, ``Fearing that the United States might annex the area, Prime Minister Macdonald acted. He had little choice but to negotiate with the M\'{e}tis.''\footnote{\cite{Francis2012},38} 
He did have another option to choose. An alternative to negotiating with the M\'{e}tis would have been to begin dismantling confederation which would offer the west to America.

%------------------------------------------------


On the horizon of a botched confederation, as Macdonald knew very well, was American annexation. 
Manifest Destiny was the order of the day south of the 49th paralel and they were already fighting the indians for lands west of their colonies.
The west was the future of Canada's prosperity and Macdonald had to claim it unambiguously. 
Red River was the frontier and Macdonald had to ``ensure it did not fall into American Hands.''\footnote{Ibid,40}
While pre-confederation British North America enjoyed reciprocity with America, this arrangement was terminated on the eve of confederation.
Twelve years after it was signed ``The Reciprocity Treaty of 1854 would terminate in 1866'' \footnote{Ibid,10}
The abrogated treaty was an ominous sign amid much talk in America about annexing the north. 
The threat of annexation was the ultimate challenge that Macdonald faced after confederation and dictated the course of action he would take on domestic issues of sovereignty.
%----------------------------------------------------------------------------------------

In Macdonald's twilight years, confederation was still being questioned.  
Confederate polititions were admonished by secessionists in the maritimes.
In Nova Scotia, ``The anti-confederate sentiments of the 1860's re-emerged.''\footnote{Ibid,95}
The dissent resulted in political action and a new Liberal premier gave body to the secessionist notion and ``introduced a secessionist resolution in the Nova Scotia legislature in 1886''\footnote{Ibid,95}
Though the legislation was fruitless, the threats did not fall on deaf ears and Macdonald's government took them seriously enough to make economic concessions. 
One more turn of the screw and Canada may have lost a province and Macdonald would have lived to see the collapse of his nascent country.

%----------------------------------------------------------------------------------------


Macdonald saved Red River from itself thus sercuring western expansion. 
He staved of annexation and appeased maritimers halting their secessionist ambitions.
What would it have taken to confound Macdonald and ruin confederation? 
Sir John A. Macdonald weathered the political storms of his day and guided young Canada to safe harbour. 
Without him, Canada would have dwindled only to have been a footnote in American history.




%----------------------------------------------------------------------------------------
%	BIBLIOGRAPHY
%----------------------------------------------------------------------------------------
\newpage
%\nocite{*}
\bibliographystyle{chicago}


\bibliography{foobar.bib}
 


%----------------------------------------------------------------------------------------

\end{document}
